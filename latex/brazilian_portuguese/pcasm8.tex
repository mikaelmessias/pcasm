%-*- latex -*-
\chapter{Dynamic Link Libraries}

\section{Using the Window's API and Dynamic Link Libraries}

UNIX systems provide a simple C based Application Programming
Interface (API).  In contrast, Microsoft Windows\texttrademark
\hspace{0.5em} packages its API in Dynamic Link Libraries that load in
each executables address space before its use. From the C/C++
programmers perspective, it appears as a normal C function call;
however, at the assembly level, it is different.

\subsection{Standard Call Calling Convention}
The Windows API uses the \emph{Standard Call}\index{calling
convention!standard call} calling convention. As stated eariler, this
convention pushes the arguments in reverse order just as the standard
C calling convention. However, there are two important
differences. First, the subroutine is responsible for clearing the
parameters from the stack. Secondly, the label of the function is
generated differently. An underscore is prepended the name as before,
but in addition the \emph{@} character is added to the end of the
function name along a number equal to the number of bytes on the stack
for the parameters of the function (in decimal).

\index{CloseHandle|(}
For example, consider the {\code CloseHandle} Windows API function. It's prototype
looks like:
\begin{lstlisting}[stepnumber=0]{}
BOOL WINAPI CloseHandle( HANDLE hObject );
\end{lstlisting}
Since a {\code HANDLE} is a double word in 32-bit Windows, the label
for this function would be {\code \_CloseHandle@4}. The {\code WINAPI}\index{WINAPI}
in the prototype is a C macro defined to be {\code \_\_stdcall}. Below
is a sample call to the function assuming that the {\code hObject}
value is in {\code EBX}.
\begin{AsmCodeListing}[frame=single]
  push  ebx             ; Push hObject on stack
  call  _CloseHandle@4
  mov   esi, eax        ; Save return value in ESI
\end{AsmCodeListing}
The stack does not have to adjusted after the function call, {\code CloseHandle}
fixes the stack automatically.
\index{CloseHandle|)}

\subsection{Static and Shared Libraries}

\index{static library|(}
A \emph{static library} is a collection of object files that can be
linked to an executable when it is created. The object code is
inserted directly into the executable just like an ordinary object
file. The library file is just a convenience. It allows a single file
to be included in the link step of the build process. All the object
code is probably not stored in the final executable. The linker will
look at which object modules are needed and only include the required
ones. Static libraries are created using the {\code LIB} program under
Windows or the {\code ar} program under UNIX. Windows libraries end in
a {\code .lib} extension and UNIX libraries end in a {\code .a}
extension.
\index{static library|)}

\index{shared library|(}
A shared library (or DLL in Windows) as its name implies shares code
among executables. When the executable is run, the OS finds all the 
shared libraries it requires and loads them into the processes memory
so the executable can use the code in them. Using the virtual memory
mapping capabilities on a protected mode operating system, shared
library code used by two or more concurrently running processes is
only loaded into physical memory once. 

There are advantages and disadvantages to shared libraries. The first
advantage is that executable sizes can be greatly reduced. If a large
library is used by many executables. Only one copy of the code (in the
shared library) is on the system (and in physical memory). If the
library was included statically, each executable would include a copy
of the code.

Another advantage is that if a bug is found in the shared library,
then the library can be replaced with one with the bug fixed. The
executables will automatically use the new fixed code without
recompiling the executable. However, this only works if the interface
of the functions in the shared library remain unchanged.

Shared libraries are also used to allow code written in different
languages to interoperate. For example, C++ can be interfaced to
Visual Basic\index{Visual Basic}, DotNet\index{DotNet} and
Java\index{Java} using shared libraries under Windows.

The disadvantages of shared libraries are that they can be more complicated
to maintain. One of the most common problems is that a new version of the
library breaks older code because it behaves slightly differently. Then
some executables need one version and others need a different one. In Windows,
this condition is known as \emph{DLL hell}\index{DLL hell}.
\index{shared library|)}

\index{DLL|(}
\subsection{Windows DLLs}

A Windows DLL is constructed much like an executable program. Object files are linked
together to create a DLL file. Unlike an executable, a DLL can have many entry points.
An entry point is just a function that can be called externally to the DLL. Only
functions that have been \emph{exported}\index{export} can be called externally.

A function (or global variable) may be exported by either entering its
name in the definition file\index{DLL!definition file} for the DLL or
by using Microsoft specific keywords.

\begin{figure}[t]
\begin{Verbatim}[frame=single,commandchars=\\\{\}]
LIBRARY \textit{library root name}
DESCRIPTION '\textit{short text description}'
EXPORTS
\textit{list of exported functions (one per line)}
\end{Verbatim}
\caption{Windows DLL definition file\label{fig:DefFile}}
\end{figure}

\subsubsection{Definition file}
This is a text file with a {\code .def} extension that lists all the
functions that the DLL exports. This file is used during the link step
of the DLL creation process. Figure~\ref{fig:DefFile} shows the
general format of a definition file.


\index{DLL|)}
