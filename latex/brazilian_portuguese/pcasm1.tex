% -*-latex-*-
\chapter{Introdu\c{c}\~ao}
\section{Sistemas Num\'ericos}

Memory in a computer consists of numbers. Computer memory does not
store these numbers in decimal (base 10). Because it greatly
simplifies the hardware, computers store all information in a binary
(base 2) format. First let's review the decimal system.

\subsection{Decimal\index{decimal}}

Base 10 numbers are composed of 10 possible digits (0-9). Each digit
of a number has a power of 10 associated with it based on its position
in the number. For example:
\begin{displaymath}
234 = 2 \times 10^2 + 3 \times 10^1 + 4 \times 10^0
\end{displaymath}

\subsection{Bin\'ario\index{bin\'ario|(}}

Base 2 numbers are composed of 2 possible digits (0 and 1). Each digit
of a number has a power of 2 associated with it based on its position
in the number. (A single binary digit is called a bit.) For
example\footnote{The 2 subscript is used to show that the number is
represented in binary, not decimal}:
\begin{eqnarray*}
11001_2 & = & 1 \times 2^4 + 1 \times 2^3 + 0 \times 2^2 + 0 \times 2^1 
              + 1 \times 2^0 \\
 & = & 16 + 8 + 1 \\
 & = & 25 
\end{eqnarray*}

This shows how binary may be converted to decimal. Table~\ref{tab:dec-bin}
shows how the first few numbers are represented in binary.
\begin{table}[t]
\begin{center}
\begin{tabular}{||c|c||cc||c|c||}
\hline
Decimal & Binary & & & Decimal & Binary \\
\hline
0       & 0000   & & & 8       & 1000 \\
\hline
1       & 0001   & & & 9       & 1001 \\
\hline
2       & 0010   & & & 10      & 1010 \\
\hline
3       & 0011   & & & 11      & 1011 \\
\hline
4       & 0100   & & & 12      & 1100 \\
\hline
5       & 0101   & & & 13      & 1101 \\
\hline
6       & 0110   & & & 14      & 1110 \\
\hline
7       & 0111   & & & 15      & 1111 \\
\hline
\end{tabular}
\caption{Decimal 0 to 15 in Binary\label{tab:dec-bin}}
\end{center}
\end{table}


\begin{figure}[h]
\begin{center}
\begin{tabular}{|rrrrrrrrp{.1cm}|p{.1cm}rrrrrrrr|}
\hline
& \multicolumn{7}{c}{No previous carry} & & & \multicolumn{7}{c}{Previous carry} & \\
\hline
&  0 & &  0 & &  1 & &  1 & & &  0 & &  0 & &  1 & & 1  & \\
& +0 & & +1 & & +0 & & +1 & & & +0 & & +1 & & +0 & & +1 &  \\
\cline{2-2} \cline{4-4} \cline{6-6} \cline{8-8} \cline{11-11} \cline{13-13} \cline{15-15} \cline{17-17}
& 0  & & 1  & & 1  & & 0  & & & 1  & & 0  & & 0  & & 1 & \\
&    & &    & &    & & c  & & &    & & c  & & c  & & c & \\
\hline
\end{tabular}

\caption{Binary addition (c stands for \emph{carry})\label{fig:bin-add}}
%TODO: Change this so that it is clear that single bits are being added,
%not 4-bit numbers. Ideas: Table or do sums horizontally.
\index{binary!addition}
\end{center}
\end{figure}

Figure~\ref{fig:bin-add} shows how individual binary digits ({\em
i.e.}, bits) are added. Here's an example:

\begin{tabular}{r}
 $11011_2$ \\
+$10001_2$ \\
\hline
$101100_2$ \\
\end{tabular}

If one considers the following decimal division:
\[ 1234 \div 10 = 123\; r\; 4 \]
he can see that this division strips off the rightmost decimal digit of 
the number and shifts the other decimal digits one position to the right.
Dividing by two performs a similar operation, but for the binary digits
of the number. Consider the following binary division:
\[ 1101_2 \div 10_2 = 110_2\; r\; 1 \]
This fact can be used to convert a decimal
number to its equivalent binary representation as
Figure~\ref{fig:dec-convert} shows. This method finds the rightmost
digit first, this digit is called the \emph{least significant bit} (lsb). 
The leftmost digit is called the \emph{most significant bit} (msb). 
The basic unit of memory consists of 8 bits and is called a \emph{byte}.
\index{binary|)}

\begin{figure}[t]
\centering
\fbox{\parbox{\textwidth}{
\begin{eqnarray*}
\mathrm{Decimal} & \mathrm{Binary} \\
25 \div 2 = 12\;r\;1 & 11001 \div 10 = 1100\;r\;1 \\
12 \div 2 = 6\;r\;0  & 1100 \div 10 = 110\;r\;0 \\
6 \div 2 = 3\;r\;0   & 110 \div 10 = 11\;r\;0 \\
3 \div 2 = 1\;r\;1   & 11 \div 10 = 1\;r\;1 \\
1 \div 2 = 0\;r\;1   & 1 \div 10 = 0\;r\;1 \\
\end{eqnarray*}

\centering
Thus $25_{10} = 11001_{2}$
}}
\caption{Decimal conversion \label{fig:dec-convert}}
\end{figure}

\subsection{Hexadecimal\index{hexadecimal|(}}

Hexadecimal numbers use base 16. Hexadecimal (or \emph{hex} for short) can be
used as a shorthand for binary numbers. Hex has 16 possible
digits. This creates a problem since there are no symbols to use for
these extra digits after 9. By convention, letters are used for these
extra digits. The 16 hex digits are 0-9 then A, B, C, D, E and F. The
digit A is equivalent to 10 in decimal, B is 11, etc. Each digit of a
hex number has a power of 16 associated with it. Example:
\begin{eqnarray*}
\rm
2BD_{16} & = & 2 \times 16^2 + 11 \times 16^1 + 13 \times 16^0 \\
         & = & 512 + 176 + 13 \\
         & = & 701 \\
\end{eqnarray*}
To convert from decimal to hex, use the same idea that was used for
binary conversion except divide by 16. See Figure~\ref{fig:hex-conv} for
an example.

\begin{figure}[t]
\centering
\fbox{\parbox{\textwidth}{
\begin{eqnarray*}
589 \div 16 & = & 36\;r\;13 \\
36 \div 16 & = & 2\;r\;4 \\
2 \div 16 & = & 0\;r\;2 \\
\end{eqnarray*}

\centering
Thus $589 = 24\mathrm{D}_{16}$
}}
\caption{\label{fig:hex-conv}}
\end{figure}

The reason that hex is useful is that there is a very simple way to
convert between hex and binary. Binary numbers get large and
cumbersome quickly. Hex provides a much more compact way to represent
binary.

To convert a hex number to binary, simply convert each hex digit to a
4-bit binary number. For example, $\mathrm{24D}_{16}$ is converted to
\mbox{$0010\;0100\; 1101_2$}. Note that the leading zeros of the
4-bits are important! If the leading zero for the middle digit of
$\mathrm{24D}_{16}$ is not used the result is wrong. Converting from
binary to hex is just as easy. One does the process in reverse. Convert
each 4-bit segments of the binary to hex. Start from the
right end, not the left end of the binary number. This ensures that
the process uses the correct 4-bit segments\footnote{If it is not
clear why the starting point makes a difference, try converting the
example starting at the left.}. Example:\newline

\begin{tabular}{cccccc}
$110$ & $0000$ & $0101$ & $1010$ & $0111$ & $1110_2$ \\
  $6$ & $0$    &   $5$  &   A  &  $7$   &    $\mathrm{E}_{16}$ \\
\end{tabular}\newline

A 4-bit number is called a \emph{nibble} \index{nibble}. Thus each hex
digit corresponds to a nibble. Two nibbles make a byte and so a byte
can be represented by a 2-digit hex number. A byte's value ranges from
0 to 11111111 in binary, 0 to FF in hex and 0 to 255 in decimal.
\index{hexadecimal|)}

\section{Organiza\c{c}\~ao do Computador}

\subsection{Mem\'oria\index{mem\'oria|(}}

The basic unit of memory is a byte. \index{byte} \MarginNote{Memory is
measured in units of kilobytes (~$2^{10} = 1,024$ bytes), megabytes
(~$2^{20} = 1,048,576$ bytes) and gigabytes (~$2^{30} = 1,073,741,824$
bytes).}A computer with 32 megabytes of memory can hold roughly 32
million bytes of information. Each byte in memory is labeled by a
unique number known as its address as Figure~\ref{fig:memory} shows.

\begin{figure}[ht]
\begin{center}
\begin{tabular}{rcccccccc}
Address & 0 & 1 & 2 & 3 & 4 & 5 & 6 & 7 \\
\cline{2-9}
Memory & \multicolumn{1}{|c}{2A}  & \multicolumn{1}{|c}{45}  
       & \multicolumn{1}{|c}{B8} & \multicolumn{1}{|c}{20} 
       & \multicolumn{1}{|c}{8F} & \multicolumn{1}{|c}{CD} 
       & \multicolumn{1}{|c}{12} & \multicolumn{1}{|c|}{2E} \\
\cline{2-9}
\end{tabular}
\caption{ Memory Addresses \label{fig:memory} }
\end{center}
\end{figure}

\begin{table}
\begin{center}
\begin{tabular}{|l|l|}
\hline
word & 2 bytes \\ \hline
double word & 4 bytes \\ \hline
quad word & 8 bytes \\ \hline
paragraph & 16 bytes \\ \hline
\end{tabular}
\caption{ Units of Memory \label{tab:mem_units} }
\end{center}
\end{table}

Often memory is used in larger chunks than single bytes. On
the PC architecture, names have been given to these larger sections of
memory as Table~\ref{tab:mem_units} shows.

All data in memory is numeric. Characters are stored by using a
\emph{character code} that maps numbers to characters. One of the 
most common character codes is known as \emph{ASCII} (American
Standard Code for Information Interchange). A new, more complete code
that is supplanting ASCII is Unicode. One key difference between the
two codes is that ASCII uses one byte to encode a character, but
Unicode uses multiple bytes. There are several different forms of Unicode.
On x86 C/C++ compilers, Unicode is represented in code using the 
{\code wchar\_t} type and the UTF-16 encoding which uses 16 bits (or a 
\emph{word}) per character. For example, ASCII maps the byte $41_{16}$ 
($65_{10}$) to the character capital \emph{A}; UTF-16 maps it to the 
word $0041_{16}$. Since ASCII uses a byte, it is limited to only 256 
different characters\footnote{In fact, ASCII only uses the lower 7-bits 
and so only has 128 different values to use.}. Unicode extends the ASCII 
values and allows many more characters to be represented. This is important 
for representing characters for all the languages of the world.
\index{memory|)}

\subsection{A CPU\index{CPU|(}}

The Central Processing Unit (CPU) is the physical device that performs
instructions. The instructions that CPUs perform are generally very
simple. Instructions may require the data they act on to be in special
storage locations in the CPU itself called
\emph{registers}. \index{register} The CPU can access data in registers
much faster than data in memory. However, the number of registers in a
CPU is limited, so the programmer must take care to keep only
currently used data in registers.

The instructions a type of CPU executes make up the CPU's
\emph{machine language}. \index{machine language} Machine programs
have a much more basic structure than higher-level languages. Machine
language instructions are encoded as raw numbers, not in friendly text
formats. A CPU must be able to decode an instruction's purpose very
quickly to run efficiently. Machine language is designed with this
goal in mind, not to be easily deciphered by humans. Programs written
in other languages must be converted to the native machine language of
the CPU to run on the computer. A \emph{compiler} \index{compiler} is
a program that translates programs written in a programming language
into the machine language of a particular computer architecture. In
general, every type of CPU has its own unique machine language. This
is one reason why programs written for a Mac can not run on an
IBM-type PC.

Computers use a \emph{clock} \index{clock} to synchronize the
execution of the \MarginNote{\emph{GHz} stands for gigahertz or one
billion cycles per second.  A 1.5 GHz CPU has 1.5 billion clock pulses
per second.} instructions.  The clock pulses at a fixed frequency
(known as the \emph{clock speed}). When you buy a 1.5 GHz computer,
1.5 GHz is the frequency of this clock\footnote{Actually, clock pulses
are used in many different components of a computer. The other
components often use different clock speeds than the CPU.}. The clock
does not keep track of minutes and seconds. It simply beats at a
constant rate. The electronics of the CPU uses the beats to perform
their operations correctly, like how the beats of a metronome help one
play music at the correct rhythm.  The number of beats (or as they are
usually called \emph{cycles}) an instruction requires depends on the
CPU generation and model. The number of cycles depends on the
instructions before it and other factors as well.


\subsection{A fam\'ilia de CPUs 80x86\index{CPU!80x86}}

IBM-type PC's contain a CPU from Intel's 80x86 family (or a clone of one).
The CPU's in this family all have some common features including a base
machine language. However, the more recent members greatly enhance the
features.
\begin{description}

\item[8088,8086:] These CPU's from the programming standpoint are
identical. They were the CPU's used in the earliest PC's. They provide
several 16-bit registers: AX, BX, CX, DX, SI, DI, BP, SP, CS, DS, SS,
ES, IP, FLAGS. They only support up to one megabyte of memory and only
operate in \emph{real mode}.  In this mode, a program may access any
memory address, even the memory of other programs! This makes
debugging and security very difficult! Also, program memory has to be
divided into \emph{segments}. Each segment can not be larger than
64K.

\item[80286:] This CPU was used in AT class PC's. It adds some new
instructions to the base machine language of the 8088/86.  However,
its main new feature is \emph{16-bit protected mode}.  In this mode,
it can access up to 16 megabytes and protect programs from accessing
each other's memory. However, programs are still divided into
segments that could not be bigger than 64K.

\item[80386:] This CPU greatly enhanced the 80286. First, it extends many
of the registers to hold 32-bits (EAX, EBX, ECX, EDX, ESI, EDI, EBP, ESP,
EIP) and adds two new 16-bit registers FS and GS. It also adds a new
\emph{32-bit protected mode}. In this mode, it can access up to 4 gigabytes.
Programs are again divided into segments, but now each segment can also be
up to 4 gigabytes in size!

\item[80486/Pentium/Pentium Pro:] These members of the 80x86 family add
very few new features. They mainly speed up the execution of the
instructions.

\item[Pentium MMX:] This processor adds the MMX (MultiMedia eXtensions)
instructions to the Pentium. These instructions can speed up common graphics
operations.

\item[Pentium II:] This is the Pentium Pro processor with the MMX instructions
added. (The Pentium III is essentially just a faster Pentium II.)

\end{description}
\index{CPU|)}

\subsection{Registradores 8086 de 16-bits\index{registrador|(}}

The original 8086 CPU provided four 16-bit general purpose registers:
AX, BX, CX and DX. Each of these registers could be decomposed into
two 8-bit registers. For example, the AX register could be decomposed
into the AH and AL registers as Figure~\ref{fig:AX_reg} shows. The AH
register contains the upper (or high) 8 bits of AX and AL contains the
lower 8 bits of AX. Often AH and AL are used as independent one byte
registers; however, it is important to realize that they are not
independent of AX. Changing AX's value will change AH and AL and
{\em vice versa}\/. The general purpose registers are used in many of
the data movement and arithmetic instructions.

\begin{figure}
\begin{center}
\begin{tabular}{cc}
\multicolumn{2}{c}{AX} \\
\hline
\multicolumn{1}{||c|}{AH} & \multicolumn{1}{c||}{AL} \\
\hline
\end{tabular}
\caption{The AX register \label{fig:AX_reg} }
\end{center}
\end{figure}

There are two 16-bit index registers\index{register!index}: SI and
DI. They are often used as pointers, but can be used for many of the
same purposes as the general registers. However, they can not be
decomposed into 8-bit registers.

The 16-bit BP and SP registers are used to point to data in the
machine language stack and are called the Base Pointer\index{register!base pointer} 
and Stack Pointer\index{register!stack pointer}, respectively. These will be discussed later. 

The 16-bit CS, DS, SS and ES registers are \emph{segment
registers}. \index{register!segment} They denote what memory is used
for different parts of a program. CS stands for Code Segment, DS for
Data Segment, SS for Stack Segment and ES for Extra Segment. ES is
used as a temporary segment register. The details of these registers
are in Sections~\ref{real_mode} and \ref{16prot_mode}.

The Instruction Pointer (IP) \index{register!IP} register is used with
the CS register to keep track of the address of the next instruction
to be executed by the CPU. Normally, as an instruction is executed, IP
is advanced to point to the next instruction in memory.

The FLAGS \index{register!FLAGS} register stores important information
about the results of a previous instruction. These results are stored
as individual bits in the register. For example, the Z bit is 1 if the
result of the previous instruction was zero or 0 if not zero. Not all
instructions modify the bits in FLAGS, consult the table in the
appendix to see how individual instructions affect the FLAGS register.

\subsection{Registradores 8086 de 32-bits\index{registrador!32-bits}}

The 80386 and later processors have extended registers. For example, the
16-bit AX register is extended to be 32-bits. To be backward compatible, AX
still refers to the 16-bit register and EAX is used to refer to the extended
32-bit register. AX is the lower 16-bits of EAX just as AL is the lower 8-bits
of AX (and EAX). There is no way to access the upper 16-bits of EAX directly.
The other extended registers are EBX, ECX, EDX, ESI and EDI. 

Many of the other registers are extended as well. BP becomes
EBP\index{register!base pointer}; SP becomes ESP\index{register!stack
pointer}; FLAGS becomes EFLAGS\index{register!EFLAGS} and IP becomes
EIP\index{register!EIP}. However, unlike the index and general purpose
registers, in 32-bit protected mode (discussed below) only the
extended versions of these registers are used.

The segment registers are still 16-bit in the 80386. There are also
two new segment registers: FS and GS\index{register!segment}. Their
names do not stand for anything. They are extra temporary segment
registers (like ES).

One of definitions of the term \emph{word} \index{word} refers to the
size of the data registers of the CPU. For the 80x86 family, the term
is now a little confusing. In Table~\ref{tab:mem_units}, one sees that
\emph{word} is defined to be 2 bytes (or 16 bits). It was given this
meaning when the 8086 was first released. When the 80386 was
developed, it was decided to leave the definition of \emph{word}
unchanged, even though the register size changed.
\index{register|)}

\subsection{Modo Real \label{modo_real} \index{modo real|(}}

In \MarginNote{So where did the infamous DOS 640K limit come from? The BIOS 
required some of the 1M for its code and for hardware devices like the video 
screen.} real mode, memory is limited to only one megabyte ($2^{20}$ bytes). 
Valid
address range from (in hex) 00000 to FFFFF.\@  % \@ means end of sentence
These addresses require a 20-bit number. Obviously, a 20-bit number will not
fit into any of the 8086's 16-bit registers. Intel solved this problem, by
using two 16-bit values to determine an address. The first 16-bit value is called
the \emph{selector}. Selector values must be stored in segment registers. The
second 16-bit value is called the \emph{offset}. The physical address 
referenced by a 32-bit \emph{selector:offset} pair is computed by the formula
\[ 16 * {\rm selector} + {\rm offset} \]
Multiplying by 16 in hex is easy, just add a 0 to the right of the number. For
example, the physical addresses referenced by 047C:0048 is given by:
\begin{center}
\begin{tabular}{r}
047C0 \\
+0048 \\
\hline
04808 \\
\end{tabular}
\end{center}
In effect, the selector value is a paragraph number
(see Table~\ref{tab:mem_units}).

Real segmented addresses have disadvantages:
\begin{itemize}
\item A single selector value can only reference 64K of memory (the
upper limit of the 16-bit offset). What if a program has more than 64K
of code? A single value in CS can not be used for the entire execution
of the program.  The program must be split up into sections (called
\emph{segments}\index{memory!segments}) less than 64K in size. When
execution moves from one segment to another, the value of CS must be
changed. Similar problems occur with large amounts of data and the DS
register. This can be very awkward!

\item Each byte in memory does not have a unique segmented address. The 
physical address 04808 can be referenced by 047C:0048, 047D:0038, 047E:0028
or 047B:0058.\@ This can complicate the comparison of segmented addresses.

\end{itemize}
\index{real mode|)}

\subsection{Modo Protegido de 16-bits \label{16modo_prot} \index{modo protegido!16-bits|(}}

In the 80286's 16-bit protected mode, selector values are interpreted
completely differently than in real mode. In real mode, a selector
value is a paragraph number of physical memory. In protected mode, a
selector value is an \emph{index} into a \emph{descriptor table}. In
both modes, programs are divided into
segments\index{memory:segments}. In real mode, these segments are at
fixed positions in physical memory and the selector value denotes the
paragraph number of the beginning of the segment. In protected mode,
the segments are not at fixed positions in physical memory. In fact,
they do not have to be in memory at all!

Protected mode uses a technique called \emph{virtual memory}
\index{memory!virtual}. The basic idea of a virtual memory system is
to only keep the data and code in memory that programs are currently
using. Other data and code are stored temporarily on disk until they
are needed again.  In 16-bit protected mode, segments are moved
between memory and disk as needed. When a segment is returned to
memory from disk, it is very likely that it will be put into a
different area of memory that it was in before being moved to
disk. All of this is done transparently by the operating system. The
program does not have to be written differently for virtual memory to
work.

In protected mode, each segment is assigned an entry in a descriptor table.
This entry has all the information that the system needs to know about the
segment. This information includes: is it currently in memory; if in memory,
where is it; access permissions ({\em e.g.\/}, read-only). The index of the
entry of the segment is the selector value that is stored in segment registers.

One \MarginNote{One well-known PC columnist called the 286 CPU ``brain
dead.''} big disadvantage of 16-bit protected mode is that offsets
are still 16-bit quantities. As a consequence of this, segment sizes
are still limited to at most 64K. This makes the use of large arrays
problematic!
\index{protected mode!16-bit|)}

\subsection{Modo Protegido de 32-bits\index{modo protegido!32-bits|(}}

The 80386 introduced 32-bit protected mode. There are two major differences
between 386 32-bit and 286 16-bit protected modes:
\begin{enumerate}
\item

Offsets are expanded to be 32-bits. This allows an offset to range up
to 4 billion. Thus, segments can have sizes up to 4 gigabytes.

\item

Segments\index{memory!segments} can be divided into smaller 4K-sized
units called \emph{pages}\index{memory!pages}. The virtual
memory\index{memory!virtual} system works with pages now instead of
segments. This means that only parts of segment may be in memory at
any one time. In 286 16-bit mode, either the entire segment is in
memory or none of it is. This is not practical with the larger
segments that 32-bit mode allows.

\end{enumerate}

\index{protected mode!32-bit|)}

In Windows 3.x, \emph{standard mode} referred to 286 16-bit protected mode and
\emph{enhanced mode} referred to 32-bit mode. Windows 9X, Windows NT/2000/XP, OS/2
and Linux all run in paged 32-bit protected mode.

\subsection{Interrup\c{c}\~oes\index{interrup\c{c}\~oes}}

Sometimes the ordinary flow of a program must be interrupted to process events
that require prompt response. The hardware of a computer provides a mechanism
called \emph{interrupts} to handle these events. For example, when a mouse is
moved, the mouse hardware interrupts the current program to handle the mouse
movement (to move the mouse cursor, {\em etc.\/}) Interrupts cause control to
be passed to an \emph{interrupt handler}. Interrupt handlers are routines that
process the interrupt. Each type of interrupt is assigned an integer number.
At the beginning of physical memory, a table of \emph{interrupt vectors}
resides that contain the segmented addresses of the interrupt handlers. The
number of interrupt is essentially an index into this table.

External interrupts are raised from outside the CPU. (The mouse is an
example of this type.) Many I/O devices raise interrupts ({\em
e.g.\/}, keyboard, timer, disk drives, CD-ROM and sound
cards). Internal interrupts are raised from within the CPU, either
from an error or the interrupt instruction. Error interrupts are also
called \emph{traps}. Interrupts generated from the interrupt
instruction are called \emph{software interrupts}. DOS uses these types of
interrupts to implement its API (Application Programming Interface). More
modern operating systems (such as Windows and UNIX) use a C based interface.
\footnote{However, they may use a lower level interface at the kernel level.}

Many interrupt handlers return control back to the interrupted program
when they finish. They restore all the registers to the same values
they had before the interrupt occurred. Thus, the interrupted program
runs as if nothing happened (except that it lost some CPU
cycles). Traps generally do not return. Often they abort the program.

\section{Linguagem de Montagem}

\subsection{Linguagem de m\'aquina\index{linguagem de m\'aquina}}

Every type of CPU understands its own machine language. Instructions
in machine language are numbers stored as bytes in memory. Each
instruction has its own unique numeric code called its \emph{operation
code} or \emph{opcode} \index{opcode} for short. The 80x86 processor's
instructions vary in size.  The opcode is always at the beginning of
the instruction. Many instructions also include data ({\em e.g.\/},
constants or addresses) used by the instruction.

Machine language is very difficult to program in directly. Deciphering the
meanings of the numerical-coded instructions is tedious for humans. For
example, the instruction that says to add the EAX and EBX registers together
and store the result back into EAX is encoded by the following hex codes:
\begin{quote}
   03 C3
\end{quote}
This is hardly obvious. Fortunately, a program called an
\emph{assembler} \index{assembler} can do this tedious work for the
programmer.

\subsection{Linguagem de montagem\index{linguagem de montagem|(}}

An assembly language program is stored as text (just as a higher level language
program). Each assembly instruction represents exactly one machine instruction.
For example, the addition instruction described above would be represented
in assembly language as:
\begin{CodeQuote}
   add eax, ebx
\end{CodeQuote}
Here the meaning of the instruction is \emph{much} clearer than in
machine code. The word {\code add} is a \emph{mnemonic}
\index{mnemonic} for the addition instruction.  The general form of an
assembly instruction is:
\begin{CodeQuote}
  {\em mnemonic operand(s)}
\end{CodeQuote}

An \emph{assembler} \index{assembler} is a program that reads a text
file with assembly instructions and converts the assembly into machine
code.  \emph{Compilers} \index{compiler} are programs that do similar
conversions for high-level programming languages. An assembler is much
simpler than a compiler. \MarginNote{It took several years for
computer scientists to figure out how to even write a compiler!} Every
assembly language statement directly represents a single machine
instruction. High-level language statements are \emph{much} more
complex and may require many machine instructions.

Another important difference between assembly and high-level languages is that
since every different type of CPU has its own machine language, it also has
its own assembly language. Porting assembly programs between different computer
architectures is \emph{much} more difficult than in a high-level language.

This book's examples uses the Netwide Assembler or NASM \index{NASM}
for short. It is freely available off the Internet (see the preface
for the URL). More common assemblers are Microsoft's Assembler (MASM)
\index{MASM} or Borland's Assembler (TASM). \index{TASM} There are
some differences in the assembly syntax for MASM/TASM and NASM.

\subsection{Operadores de instru\c{c}\~ao}

Machine code instructions have varying number and type of operands; however,
in general, each instruction itself will have a fixed number of operands (0
to 3). Operands can have the following types:
\begin{description}
\item[register:]
These operands refer directly to the contents of the CPU's registers.
\item[memory:]
These refer to data in memory. The address of the data may be a constant
hardcoded into the instruction or may be computed using values of registers.
Address are always offsets from the beginning of a segment.
\item[immediate:]
\index{immediate}
These are fixed values that are listed in the instruction itself. They are
stored in the instruction itself (in the code segment), not in the data
segment.
\item[implied:]
These operands are not explicitly shown. For example, the increment 
instruction adds one to a register or memory. The one is implied.
\end{description}
\index{assembly language|)}

\subsection{Instru\c{c}\~oes b\'asicas}

The most basic instruction is the {\code MOV} \index{MOV} instruction. It moves
data from one location to another (like the assignment operator in a
high-level language). It takes two operands:
\begin{CodeQuote}
  mov {\em dest, src}
\end{CodeQuote}
The data specified by {\em src} is copied to {\em dest\/}. One restriction
is that both operands may not be memory operands. This points out another
quirk of assembly. There are often somewhat arbitrary rules about how the
various instructions are used. The operands must also be the same size. The
value of AX can not be stored into BL.

Here is an example (semicolons start a comment\index{comment}):
\begin{AsmCodeListing}[frame=none, numbers=none]
      mov    eax, 3   ; store 3 into EAX register (3 is immediate operand) 
      mov    bx, ax   ; store the value of AX into the BX register 
\end{AsmCodeListing}

The {\code ADD} \index{ADD} instruction is used to add integers.
\begin{AsmCodeListing}[frame=none, numbers=none]
      add    eax, 4   ; eax = eax + 4
      add    al, ah   ; al = al + ah 
\end{AsmCodeListing}

The {\code SUB} \index{SUB} instruction subtracts integers.
\begin{AsmCodeListing}[frame=none, numbers=none]
      sub    bx, 10   ; bx = bx - 10
      sub    ebx, edi ; ebx = ebx - edi
\end{AsmCodeListing}

The {\code INC} \index{INC} and {\code DEC} \index{DEC} instructions
increment or decrement values by one. Since the one is an implicit
operand, the machine code for {\code INC} and {\code DEC} is smaller
than for the equivalent {\code ADD} and {\code SUB} instructions.
\begin{AsmCodeListing}[frame=none, numbers=none]
      inc    ecx      ; ecx++
      dec    dl       ; dl--
\end{AsmCodeListing}

\subsection{Diretivas\index{diretiva|(}}

A \emph{directive} is an artifact of the assembler not the CPU. They are
generally used to either instruct the assembler to do something or inform
the assembler of something. They are not translated into machine code. Common
uses of directives are:
\begin{list}{$\bullet$}{\setlength{\itemsep}{0pt}}
\item define constants
\item define memory to store data into
\item group memory into segments
\item conditionally include source code
\item include other files
\end{list}

NASM code passes through a preprocessor just like C. It has many of the
same preprocessor commands as C. However, NASM's preprocessor directives 
start with a \% instead of a \# as in C.

\subsubsection{A diretiva equ\index{diretiva!equ}}

The {\code equ} directive can be used to define a \emph{symbol}. Symbols are
named constants that can be used in the assembly program. The format is:
\begin{quote}
  \code {\em symbol} equ {\em value}
\end{quote}
Symbol values can \emph{not} be redefined later.

\subsubsection{A diretiva \%define\index{diretiva!\%define}}

This directive is similar to C's {\code \#define} directive. It is
most commonly used to define constant macros just as in C.
\begin{AsmCodeListing}[frame=none, numbers=none]
%define SIZE 100
      mov    eax, SIZE
\end{AsmCodeListing}
The above code defines a macro named {\code SIZE} and shows its use in
a {\code MOV} instruction. Macros are more flexible than symbols in
two ways. Macros can be redefined and can be more than simple constant
numbers.

\subsubsection{Diretivas de dados\index{directiva!dado|(}}

\begin{table}[t]
\centering
\begin{tabular}{||c|c||} \hline
{\bf Unit} & {\bf Letter} \\
\hline
byte & B \\
word & W \\
double word & D \\
quad word & Q \\
ten bytes & T \\
\hline
\end{tabular}
\caption{Letters for {\code RESX} and {\code DX} Directives 
         \label{tab:size-letters} }
\end{table}

Data directives are used in data segments to define room for
memory. There are two ways memory can be reserved. The first way only
defines room for data; the second way defines room and an initial
value. The first method uses one of the {\code RES{\em
X}}\index{directive!RES\emph{X}} directives. The {\em X} is replaced
with a letter that determines the size of the object (or objects) that
will be stored. Table~\ref{tab:size-letters} shows the possible
values.

The second method (that defines an initial value, too) uses one of the
{\code D{\em X}} directives\index{directive!D\emph{X}}. The {\em X}
letters are the same as those in the {\code RES{\em X}} directives.

It is very common to mark memory locations with
\emph{labels}. \index{label} Labels allow one to easily refer to
memory locations in code. Below are several examples:
\begin{AsmCodeListing}[frame=none, numbers=none]
L1    db     0        ; byte labeled L1 with initial value 0
L2    dw     1000     ; word labeled L2 with initial value 1000
L3    db     110101b  ; byte initialized to binary 110101 (53 in decimal)
L4    db     12h      ; byte initialized to hex 12 (18 in decimal)
L5    db     17o      ; byte initialized to octal 17 (15 in decimal)
L6    dd     1A92h    ; double word initialized to hex 1A92
L7    resb   1        ; 1 uninitialized byte
L8    db     "A"      ; byte initialized to ASCII code for A (65)
\end{AsmCodeListing}

Double quotes and single quotes are treated the same. Consecutive data
definitions are stored sequentially in memory. That is, the word L2 is
stored immediately after L1 in memory. Sequences of memory may also be
defined.
\begin{AsmCodeListing}[frame=none, numbers=none]
L9    db     0, 1, 2, 3              ; defines 4 bytes
L10   db     "w", "o", "r", 'd', 0   ; defines a C string = "word"
L11   db     'word', 0               ; same as L10
\end{AsmCodeListing}

The {\code DD}\index{directive!DD} directive can be used to define
both integer and single precision floating point\footnote{Single
precision floating point is equivalent to a {\code float} variable in
C.} constants. However, the {\code DQ}\index{directive!DQ} can only
be used to define double precision floating point constants.

For large sequences, NASM's {\code TIMES} \index{directive!TIMES}
directive is often useful. This directive repeats its operand a
specified number of times. For example,
\begin{AsmCodeListing}[frame=none, numbers=none]
L12   times 100 db 0                 ; equivalent to 100 (db 0)'s
L13   resw   100                     ; reserves room for 100 words
\end{AsmCodeListing}
\index{directive!data|)}
\index{directive|)}

\index{label|(}
Remember that labels  can be used to refer to data in code. There are
two ways that a label can be used. If a plain label is used, it is
interpreted as the address (or offset) of the data. If the label is
placed inside square brackets ({\code []}), it is interpreted as the data at
the address. In other words, one should think of a label as a \emph{pointer}
to the data and the square brackets dereferences the pointer just as
the asterisk does in C. (MASM/TASM follow a different convention.) 
In 32-bit mode, addresses are 32-bit. Here are some examples:
\begin{AsmCodeListing}[frame=none]
      mov    al, [L1]      ; copy byte at L1 into AL
      mov    eax, L1       ; EAX = address of byte at L1
      mov    [L1], ah      ; copy AH into byte at L1
      mov    eax, [L6]     ; copy double word at L6 into EAX
      add    eax, [L6]     ; EAX = EAX + double word at L6
      add    [L6], eax     ; double word at L6 += EAX
      mov    al, [L6]      ; copy first byte of double word at L6 into AL
\end{AsmCodeListing}
Line 7 of the examples shows an important property of NASM. The assembler does
\emph{not} keep track of the type of data that a label refers to. It is up
to the programmer to make sure that he (or she) uses a label correctly. Later
it will be common to store addresses of data in registers and use the register
like a pointer variable in C. Again, no checking is made that a pointer is
used correctly. In this way, assembly is much more error prone than even C.

Consider the following instruction:
\begin{AsmCodeListing}[frame=none, numbers=none]
      mov    [L6], 1             ; store a 1 at L6
\end{AsmCodeListing}
This statement produces an {\code operation size not specified} error. Why?
Because the assembler does not know whether to store the 1 as a byte, word
or double word. To fix this, add a size specifier:
\begin{AsmCodeListing}[frame=none, numbers=none]
      mov    dword [L6], 1       ; store a 1 at L6
\end{AsmCodeListing}
\index{DWORD}This tells the assembler to store an 1 at the double word that starts at
{\code L6}. Other size specifiers are: {\code BYTE}\index{BYTE}, {\code WORD}\index{WORD},
{\code QWORD}\index{QWORD} and {\code TWORD}\footnote{{\code TWORD} defines a ten byte
area of memory. The floating point coprocessor uses this data type.}\index{TWORD}.
\index{label|)}

\subsection{Entrada e Sa\'ida' \index{I/O|(}}

Input and output are very system dependent activities. It involves
interfacing with the system's hardware. High level languages, like C,
provide standard libraries of routines that provide a simple, uniform
programming interface for I/O.  Assembly languages provide no standard
libraries. They must either directly access hardware (which is a privileged
operation in protected mode) or use whatever low level routines that the
operating system provides.

\index{I/O!asm\_io library|(} 
It is very common for assembly routines to be interfaced with C. One
advantage of this is that the assembly code can use the standard C
library I/O routines.  However, one must know the rules for passing
information between routines that C uses. These rules are too
complicated to cover here. (They are covered later!) To simplify I/O,
the author has developed his own routines that hide the complex C
rules and provide a much more simple interface.  Table~\ref{tab:asmio}
describes the routines provided. All of the routines preserve the
value of all registers, except for the read routines. These routines
do modify the value of the EAX register. To use these routines, one
must include a file with information that the assembler needs to use
them.  To include a file in NASM, use the {\code \%include}
preprocessor directive. The following line includes the file needed by
the author's I/O routines\footnote{The {\code asm\_io.inc} (and the
{\code asm\_io} object file that {\code asm\_io.inc} requires) are in
the example code downloads on the web page for this tutorial, {\code
http://pacman128.github.io/pcasm/}}:
\begin{AsmCodeListing}[frame=none, numbers=none]
%include "asm_io.inc"
\end{AsmCodeListing}

\begin{table}[t]
\centering
\begin{tabular}{lp{3.5in}}
{\bf print\_int} & prints out to the screen the value of the integer stored 
                  in EAX \\
{\bf print\_char} & prints out to the screen the character whose
                    ASCII value stored in AL \\
{\bf print\_string} & prints out to the screen the contents of the string
                     at the {\em address} stored in EAX. The string must be
                     a C-type string ({\em i.e.} null terminated). \\
{\bf print\_nl} & prints out to the screen a new line character. \\
{\bf read\_int} & reads an integer from the keyboard and stores it into the
                 EAX register. \\
{\bf read\_char} & reads a single character from the keyboard and stores its
                  ASCII code into the EAX register. \\
\end{tabular}
\caption{Assembly I/O Routines \label{tab:asmio} \index{I/O!asm\_io library!print\_int}
\index{I/O!asm\_io library!print\_char} \index{I/O!asm\_io library!print\_string} 
\index{I/O!asm\_io library!print\_nl} \index{I/O!asm\_io library!read\_int}
\index{I/O!asm\_io library!read\_char}}
\end{table}

To use one of the print routines, one loads EAX with the correct value and
uses a {\code CALL} instruction to invoke it. The {\code CALL} instruction
is equivalent to a function call in a high level language. It jumps execution
to another section of code, but returns back to its origin after the routine
is over. The example program below shows several examples of calls to these
I/O routines.

\subsection{Depura\c{c}\~ao\index{depura\c{c}\~ao|(}}

The author's library also contains some useful routines for debugging 
programs. These debugging routines display information about the state of
the computer without modifying the state. These routines are really
\emph{macros} that preserve the current state of the CPU and then make a
subroutine call. The macros are defined in the {\code asm\_io.inc} file
discussed above. Macros are used like ordinary instructions. Operands of
macros are separated by commas.

There are four debugging routines named {\code dump\_regs}, {\code
dump\_mem}, {\code dump\_stack} and {\code dump\_math}; they display
the values of registers, memory, stack and the math coprocessor,
respectively.
\begin{description}

\item[dump\_regs]
\index{I/O!asm\_io library!dump\_regs} 
This macro prints out the values of the registers (in hexadecimal) of
the computer to {\code stdout} (\emph{i.e.} the screen). It also
displays the bits set in the FLAGS\footnote{Chapter~2 discusses this
register} register. For example, if the zero flag is 1, \emph{ZF} is
displayed. If it is 0, it is not displayed. It takes a single integer
argument that is printed out as well. This can be used to distinguish
the output of different {\code dump\_regs} commands.

\item[dump\_mem]
\index{I/O!asm\_io library!dump\_mem} 
This macro prints out the values of a region of memory (in
hexadecimal) and also as ASCII characters. It takes three comma delimited
arguments. The first is an integer that is used to label the
output (just as {\code dump\_regs} argument). The second argument is
the address to display. (This can be a label.) The last argument is
the number of 16-byte paragraphs to display after the address. The
memory displayed will start on the first paragraph boundary before the
requested address.

\item[dump\_stack]
\index{I/O!asm\_io library!dump\_stack} 
This macro prints out the values on the CPU stack. (The stack will be
covered in Chapter~4.) The stack is organized as double words and this
routine displays them this way. It takes three comma delimited
arguments. The first is an integer label (like {\code
dump\_regs}). The second is the number of double words to display
\emph{below} the address that the {\code EBP} register holds and the
third argument is the number of double words to display \emph{above}
the address in {\code EBP}.

\item[dump\_math]
\index{I/O!asm\_io library!dump\_math} 
This macro prints out the values of the registers of the math coprocessor.
It takes a single integer argument that is used to label the output just as
the argument of {\code dump\_regs} does.
\end{description}
\index{debugging|)}
\index{I/O!asm\_io library|)} 
\index{I/O|)}

\section{Criando um programa}

Today, it is unusual to create a stand alone program written
completely in assembly language. Assembly is usually used to key certain
critical routines. Why? It is \emph{much} easier to program in a higher level 
language than in assembly. Also, using assembly makes a program very hard to
port to other platforms. In fact, it is rare to use assembly at all.

So, why should anyone learn assembly at all?
\begin{enumerate}
\item Sometimes code written in assembly can be faster and smaller than
      compiler generated code.
\item Assembly allows access to direct hardware features of the system that
      might be difficult or impossible to use from a higher level language.
\item Learning to program in assembly helps one gain a deeper understanding of
      how computers work.
\item Learning to program in assembly helps one understand better how compilers
      and high level languages like C work.
\end{enumerate}
These last two points demonstrate that learning assembly can be useful even if
one never programs in it later. In fact, the author rarely programs in
assembly, but he uses the ideas he learned from it everyday.

\subsection{Primeiro programa}

\begin{figure}[t]
\begin{lstlisting}[frame=tlrb]{}
int main()
{
  int ret_status;
  ret_status = asm_main();
  return ret_status;
}
\end{lstlisting}
\caption{{\code driver.c} code\label{fig:driverProg} \index{C driver}}
\end{figure}

The early programs in this text will all start from the simple C
driver program in Figure~\ref{fig:driverProg}. It simply calls
another function named {\code asm\_main}. This is really a routine
that will be written in assembly. There are several advantages in
using the C driver routine. First, this lets the C system set up the
program to run correctly in protected mode. All the segments and their
corresponding segment registers will be initialized by C. The assembly
code need not worry about any of this. Secondly, the C library will
also be available to be used by the assembly code. The author's I/O
routines take advantage of this. They use C's I/O functions ({\code
printf}, {\em etc.}).  The following shows a simple assembly program.

\begin{AsmCodeListing}[label=first.asm]
; file: first.asm
; First assembly program. This program asks for two integers as
; input and prints out their sum.
;
; To create executable using djgpp:
; nasm -f coff first.asm
; gcc -o first first.o driver.c asm_io.o

%include "asm_io.inc"
;
; initialized data is put in the .data segment
;
segment .data
;
; These labels refer to strings used for output
;
prompt1 db    "Enter a number: ", 0       ; don't forget null terminator
prompt2 db    "Enter another number: ", 0
outmsg1 db    "You entered ", 0
outmsg2 db    " and ", 0
outmsg3 db    ", the sum of these is ", 0

;
; uninitialized data is put in the .bss segment
;
segment .bss
;
; These labels refer to double words used to store the inputs
;
input1  resd 1
input2  resd 1

;
; code is put in the .text segment
;
segment .text
        global  _asm_main
_asm_main:
        enter   0,0               ; setup routine
        pusha

        mov     eax, prompt1      ; print out prompt
        call    print_string

        call    read_int          ; read integer
        mov     [input1], eax     ; store into input1

        mov     eax, prompt2      ; print out prompt
        call    print_string

        call    read_int          ; read integer
        mov     [input2], eax     ; store into input2

        mov     eax, [input1]     ; eax = dword at input1
        add     eax, [input2]     ; eax += dword at input2
        mov     ebx, eax          ; ebx = eax

        dump_regs 1                ; print out register values
        dump_mem  2, outmsg1, 1    ; print out memory
;
; next print out result message as series of steps
;
        mov     eax, outmsg1
        call    print_string      ; print out first message
        mov     eax, [input1]     
        call    print_int         ; print out input1
        mov     eax, outmsg2
        call    print_string      ; print out second message
        mov     eax, [input2]
        call    print_int         ; print out input2
        mov     eax, outmsg3
        call    print_string      ; print out third message
        mov     eax, ebx
        call    print_int         ; print out sum (ebx)
        call    print_nl          ; print new-line

        popa
        mov     eax, 0            ; return back to C
        leave                     
        ret
\end{AsmCodeListing}

Line~13 of the program defines a section of the program that specifies
memory to be stored in the data segment (whose name is {\code
.data})\index{data segment}. Only initialized data should be defined
in this segment. On lines~17 to 21, several strings are declared. They
will be printed with the C library and so must be terminated with a
\emph{null} character (ASCII code 0).  Remember there is a big
difference between {\code 0} and {\code '0'}.

Uninitialized data should be declared in the bss segment (named {\code
.bss} on line 26)\index{bss segment}. This segment gets its name from an early UNIX-based
assembler operator that meant ``block started by symbol.'' There is
also a stack segment too. It will be discussed later.

The code segment \index{code segment} is named {\code .text}
historically. It is where instructions are placed. Note that the code
label for the main routine (line~38) has an underscore prefix.  This
is part of the \emph{C calling convention}. \index{calling
convention!C} This convention specifies the rules C uses when compiling
code. It is very important to know this convention when interfacing C
and assembly. Later the entire convention will be presented; however,
for now, one only needs to know that all C symbols ({\em i.e.},
functions and global variables) have a underscore prefix appended to
them by the C compiler. (This rule is specifically for DOS/Windows,
the Linux C compiler does not prepend anything to C symbol names.)

The {\code global} {\index{directive!global} directive on line 37
tells the assembler to make the {\code \_asm\_main} label
global. Unlike in C, labels have \emph{internal scope} by
default. This means that only code in the same module can use the
label. The {\code global} directive gives the specified label (or
labels) \emph{external scope}. This type of label can be accessed by
any module in the program. The {\code asm\_io} module declares the
{\code print\_int}, {\em et.al.\/} labels to be global. This is why
one can use them in the {\code first.asm} module.

\subsection{Depend\^encias do compilador}

The assembly code above is specific to the free GNU\footnote{GNU is a
project of the Free Software Foundation ({\code
http://www.fsf.org})}-based DJGPP \index{compiler!DJGPP} C/C++
compiler.\footnote{\code http://www.delorie.com/djgpp} This compiler
can be freely downloaded from the Internet. It requires a 386-based PC
or better and runs under DOS, Windows 95/98 or NT. This compiler uses
object files in the COFF (Common Object File Format) format. To
assemble to this format use the {\code -f~coff} switch with {\code
nasm} (as shown in the comments of the above code). The extension of
the resulting object file will be {\code o}.

The Linux C compiler is a GNU compiler also. \index{compiler!gcc} To
convert the code above to run under Linux, simply remove the
underscore prefixes in lines~37 and 38. Linux uses the ELF (Executable
and Linkable Format) format for object files. Use the {\code -f~elf}
switch for Linux. It also produces an object with an {\code o}
extension.\MarginNote{The compiler specific example files, available
from the author's web site, have already been modified to work with
the appropriate compiler.}

Borland C/C++ \index{compiler!Borland} is another popular compiler. It
uses the Microsoft OMF format for object files. Use the {\code -f~obj}
switch for Borland compilers. The extension of the object file will be
{\code obj}. The OMF format uses different {\code segment} directives
than the other object formats. The data segment (line~13) must be
changed to:
\begin{CodeQuote}
segment \_DATA public align=4 class=DATA use32
\end{CodeQuote}
The bss segment (line 26) must be changed to:
\begin{CodeQuote}
segment \_BSS public align=4 class=BSS use32
\end{CodeQuote}
The text segment (line 36) must be changed to:
\begin{CodeQuote}
segment \_TEXT public align=1 class=CODE use32
\end{CodeQuote}
In addition a new line should be added before line 36:
\begin{CodeQuote}
group DGROUP \_BSS \_DATA
\end{CodeQuote}

The Microsoft C/C++ \index{compiler!Microsoft} compiler can use either
the OMF format or the Win32 format for object files. (If given a OMF
format, it converts the information to Win32 format internally.) Win32
format allows segments to be defined just as for DJGPP and Linux. Use
the {\code -f~win32} switch to output in this mode. The extension of
the object file will be {\code obj}.

\subsection{Montando o c\'odigo}

The first step is to assemble the code. From the command line, type:
\begin{CodeQuote}
nasm -f {\em object-format} first.asm
\end{CodeQuote}
where {\em object-format} is either {\em coff\/}, {\em elf\/}, {\em obj} or
{\em win32} depending on what C compiler will be used. (Remember that the
source file must be changed for both Linux and Borland as well.)


\subsection{Compilando o c\'odigo em C}

Compile the {\code driver.c} file using a C compiler. For DJGPP, use:
\begin{CodeQuote}
gcc -c driver.c
\end{CodeQuote}
The {\code -c} switch means to just compile, do not attempt to link yet. This
same switch works on Linux, Borland and Microsoft compilers as well.

\subsection{Vinculando os arquivos de objeto \label{seq:vinculando} \index{vinculando|(}}

Linking is the process of combining the machine code and data in
object files and library files together to create an executable
file. As will be shown below, this process is complicated.

C code requires the standard C library and special \emph{startup code}
\index{startup code} to run.  It is \emph{much} easier to let the C
compiler call the linker with the correct parameters, than to try to
call the linker directly. For example, to link the code for the first
program using DJGPP, \index{compiler!DJGPP} use:
\begin{CodeQuote}
gcc -o first driver.o first.o asm\_io.o
\end{CodeQuote}
This creates an executable called {\code first.exe} (or just {\code first}
under Linux). 

With Borland, \index{compiler!Borland} one would use:
\begin{CodeQuote}
bcc32 first.obj driver.obj asm\_io.obj
\end{CodeQuote}
Borland uses the name of the first file listed to determine the executable
name. So in the above case, the program would be named {\code first.exe}.

It is possible to combine the compiling and linking step. For example,
\begin{CodeQuote}
gcc -o first {\em driver.c} first.o asm\_io.o
\end{CodeQuote}
Now {\code gcc} will compile {\code driver.c} and then link.
\index{linking|)}

\subsection{Entendendo um arquivo de listagem de montagem \index{arquivo de listagem|(}}

The {\code -l {\em listing-file}} switch can be used to tell {\code
nasm} to create a listing file of a given name. This file shows how
the code was assembled. Here is how lines~17 and 18 (in the data
segment) appear in the listing file. (The line numbers are in the
listing file; however notice that the line numbers in the source file
may not be the same as the line numbers in the listing file.)
\begin{Verbatim}[xleftmargin=\AsmMargin]
48 00000000 456E7465722061206E-     prompt1 db    "Enter a number: ", 0
49 00000009 756D6265723A2000
50 00000011 456E74657220616E6F-     prompt2 db    "Enter another number: ", 0
51 0000001A 74686572206E756D62-
52 00000023 65723A2000
 \end{Verbatim}
The first column in each line is the line number and the second is the
offset (in hex) of the data in the segment. The third column shows the
raw hex values that will be stored. In this case the hex data
correspond to ASCII codes. Finally, the text from the source file is
displayed on the line. The offsets listed in the second column are
very likely \emph{not} the true offsets that the data will be placed
at in the complete program.  Each module may define its own labels in
the data segment (and the other segments, too). In the link step (see
section~\ref{seq:linking}), all these data segment label definitions
are combined to form one data segment. The new final offsets are then
computed by the linker.

Here is a small section (lines~54 to 56 of the source file) of the
text segment in the listing file:
\begin{Verbatim}[xleftmargin=\AsmMargin]
94 0000002C A1[00000000]          mov     eax, [input1]
95 00000031 0305[04000000]        add     eax, [input2]
96 00000037 89C3                  mov     ebx, eax
\end{Verbatim}
The third column shows the machine code generated by the
assembly. Often the complete code for an instruction can not be
computed yet. For example, in line~94 the offset (or address) of
{\code input1} is not known until the code is linked. The assembler
can compute the op-code for the {\code mov} instruction (which from
the listing is A1), but it writes the offset in square brackets
because the exact value can not be computed yet. In this case, a
temporary offset of 0 is used because {\code input1} is at the
beginning of the part of the bss segment defined in this
file. Remember this does \emph{not} mean that it will be at the
beginning of the final bss segment of the program. When the code is linked,
the linker will insert the correct offset into the position. Other 
instructions, like line~96, do not reference any labels. Here the assembler
can compute the complete machine code.
\index{listing file|)}

\subsubsection{Representa\c{c}\~oes Big e Little Endian \index{endianess|(}}
If one looks closely at line~95, something seems very strange about
the offset in the square brackets of the machine code. The {\code
input2} label is at offset 4 (as defined in this file); however, the
offset that appears in memory is not 00000004, but 04000000. Why? Different
processors store multibyte integers in different orders in
memory. There are two popular methods of storing integers: \emph{big
endian} and \emph{little endian}. \MarginNote{Endian is pronounced
like \emph{indian}.} Big endian is the method that seems the most
natural. The biggest (\emph{i.e.} most significant) byte is stored
first, then the next biggest, \emph{etc.} For example, the dword
00000004 would be stored as the four bytes 00~00~00~04.  IBM
mainframes, most RISC processors and Motorola processors all use this
big endian method. However, Intel-based processors use the little
endian method! Here the least significant byte is stored first. So,
00000004 is stored in memory as 04~00~00~00. This format is hardwired
into the CPU and can not be changed. Normally, the programmer does not
need to worry about which format is used. However, there are
circumstances where it is important.
\begin{enumerate}
\item When binary data is transfered between different computers (either from
      files or through a network).
\item When binary data is written out to memory as a multibyte integer and
      then read back as individual bytes or \emph{vice versa}.
\end{enumerate}

Endianness does not apply to the order of array elements. The first
element of an array is always at the lowest address. This applies to
strings (which are just character arrays). Endianness still
applies to the individual elements of the arrays. 
\index{endianess|)}

\begin{figure}[t]
\begin{AsmCodeListing}[label=skel.asm]
%include "asm_io.inc"
segment .data
;
; initialized data is put in the data segment here
;

segment .bss
;
; uninitialized data is put in the bss segment
;

segment .text
        global  _asm_main
_asm_main:
        enter   0,0               ; setup routine
        pusha

;
; code is put in the text segment. Do not modify the code before
; or after this comment.
;

        popa
        mov     eax, 0            ; return back to C
        leave                     
        ret
\end{AsmCodeListing}
\caption{Skeleton Program \label{fig:skel}}
\end{figure}

\section{Arquivo base \index{arquivo base}}

Figure~\ref{fig:skel} shows a skeleton file that can be used as a starting
point for writing assembly programs.







