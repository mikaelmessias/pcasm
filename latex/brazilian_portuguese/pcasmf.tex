% -*-LaTex-*-
%front matter of pcasm book

\chapter{Prefácio}

\section*{Propósito}

O propósito desse livro é oferecer ao leitor uma melhor compreensão
de como computadores realmente funcionam em níveis mais baixos que o de 
linguagens como o Pascal. A produtividade durante o desenvolvimento de
softwares em linguagens de alto nível, como C e C++, pode ser muito maior
quando se obtém um conhecimento aprofudado no processo de funcionamento de um
computador. Aprender a programar em linguagem de montagem é uma excelente forma
de atingir esse objetivo. Outros livros de linguagem de montagem para 
computadores podem ainda ensinar como programar processadores 8086, o mesmo
do primeiro computador utilizado em 1981! O processador 8086 suporta apenas o modo
\emph{real}. Nesse modo, qualquer programa pode endereçar qualquer memória ou
dispositivo do computador. Esse modo não é adequado para sistemas operacionais
seguros e multitarefas. Esse livro discute, como alternativa, a programar 
processadores 80386 e posteriores em modo \emph{protegido} (o modo em que o 
Windows e o Linux são executados). Esse modo suporta os recursos que sistemas 
operacionais esperam, como memória virtual e proteção de memória.
Há diversas razões que justificam a utilização do modo protegido:

\begin{enumerate}
\item Programar em modo protegido é mais fácil do que no modo real 8086
	  que outros livros usam.
\item Todos os sistemas operacionis modernos são executados em modo protegido.
\item Há softwares livres disponíveis que são executados nesse modo.
\end{enumerate}
A falta de livros para programação assembly em modo protegido é a principal
razão pela qual o autor escreveu esse livro.

Conforme mencionado acima, esse livro faz uso de softwares de
código livre/aberto: o montador NASM, e o compilador
C/C++ DJGPP. Ambos estão disponíveis para download na internet. O livro
também discute como compilar o código de montagem NASM no Linux e com os
compiladores C/C++ da Microsoft e da Borland no Windows. Exemplos para todas
essas plataformas podem ser encontrados em meu web site:
{\code http://pacman128.github.io/pcasm/}.
é \emph{necessário} fazer download dos códigos de exemplos para montar e
executar muitos dos exemplos desse tutorial.

Tenha em mente que esse livro não tenta cobrir cada aspecto de programação
assembly. O autor tentou abordar tópicos importantes que \emph{todo} programador
deveria estar familiarizado.

\section*{Agradecimentos}

O autor gostaria de agradecer a muitos programadores ao redor do mundo
que contribuiram com o movimento do código livre/aberto. Todos os programas
e até mesmo esse livro foram produzidos utilizando software livre. Especificamente,
o autor gostaria de agradecer a John~S.~Fine, Simon~Tatham, Julian~Hall e outros por
desenvolverem o montador NASM no qual todos os exemplos desse livro foram baseados;
DJ Delorie por desenvolver o compilador C/C++ DJGPP; à uma numerosa quantidade de
pessoas que contribuiram para o compilador GNU gcc em qual o DJGPP é baseado; 
Donald Knuth e outros por desenvolverem os sistemas de tipagem \TeX\ e \LaTeXe\ que
foram usados para produzir esse livro; Richard Stallman (fundador da Fundação de
Software Livre), Linus Torvalds (criador do kernel Linux) e outros
que produziram softwares fundamentais usados na produção desse trabalho.

Agradeço às seguintes pessoas pelas correções:
\begin{itemize}
\item John S. Fine
\item Marcelo Henrique Pinto de Almeida
\item Sam Hopkins
\item Nick D'Imperio
\item Jeremiah Lawrence
\item Ed Beroset
\item Jerry Gembarowski
\item Ziqiang Peng
\item Eno Compton
\item Josh I Cates
\item Mik Mifflin
\item Luke Wallis
\item Gaku Ueda
\item Brian Heward
\item Chad Gorshing
\item F. Gotti
\item Bob Wilkinson
\item Markus Koegel
\item Louis Taber
\item Dave Kiddell
\item Eduardo Horowitz
\item S\'{e}bastien Le Ray
\item Nehal Mistry
\item Jianyue Wang
\item Jeremias Kleer
\item Marc Janicki
\item Trevor Hansen
\item Giacomo Bruschi
\item Leonardo Rodr\'{i}guez M\'{u}jica
\item Ulrich Bicheler
\item Wu Xing
\item Oleksandr Baranyuk
\end{itemize}

\section*{Recursos na Internet (em ingl\^es)}
\begin{center}
\begin{tabular}{|ll|}
\hline
Página do autor & {\code http://www.drpaulcarter.com/} \\
NASM SourceForge & {\code http://sourceforge.net/projects/nasm/} \\
DJGPP & {\code http://www.delorie.com/djgpp} \\
Linux Assembly & {\code http://www.linuxassembly.org/} \\
The Art of Assembly & {\code http://webster.cs.ucr.edu/} \\
USENET & {\code comp.lang.asm.x86} \\
\hline
\end{tabular}
\end{center}


\section*{Feedback}

O autor agradece qualquer feedback desse livro.
\begin{center}
\begin{tabular}{ll}
\textbf{E-mail:} & {\code pacman128@gmail.com} \\
\textbf{WWW:}    & {\code http://pacman128.github.io/pcasm/} \\
\end{tabular}
\end{center}


